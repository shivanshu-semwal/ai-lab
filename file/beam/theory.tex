Beam search is a heuristic search algorithm that explores a graph by expanding
the most promising node in a limited set. Beam search is an optimization of
best-first search that reduces its memory requirements. Best-first search is a
graph search which orders all partial solutions (states) according to some
heuristic. But in beam search, only a predetermined number of best partial
solutions are kept as candidates. It is thus a greedy algorithm.

\subsubsection*{Algorithm}

\begin{itemize}
    \item set Node = rootNode and found = False
    \item if node is goalNode then found = True
    \item else find the successors of node with their estimated
          cost and store it in the open list.
    \item while found is False and open list is not empty
          \begin{itemize}
              \item sort the open list
              \item select top w candidate from the open list and put it in wopen list.
                    Clean the open list.
              \item for each node in wopen list
                    \begin{itemize}
                        \item if node is the goal node then found = True
                        \item else find the successors of the node with their estimated
                              cost and store it in the open list.
                    \end{itemize}
          \end{itemize}
\end{itemize}

\subsubsection*{Analysis:}

\begin{itemize}
      \item $b$ - branching factor
      \item $d$ - depth of the graph
      \item $w$ - is the beam width
      \item \textbf{Time Complexity:} $O(b^d)$
      \item \textbf{Time Complexity:} $O(w)$
      \item Incomplete search
\end{itemize}