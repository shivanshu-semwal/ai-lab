Best first search is a traversal technique that decides which node is to be
visited next by checking which node is the most promising one and then check it.
For this it uses an evaluation function to decide the traversal.

This best first search technique of tree traversal comes under the category of
heuristic search or informed search technique.

The cost of nodes is stored in a priority queue. This makes implementation of
best-first search is same as that of breadth First search. We will use the
priority queue just like we use a queue for BFS.

\subsubsection*{Algorithm}

\begin{itemize}
    \item Create a priorityQueue pqueue.
    \item insert start in pqueue : pqueue.insert(start)
    \item delete all elements of pqueue one by one.
          \begin{itemize}
              \item if, the element is goal. Exit.
              \item else, traverse neighbors and mark the node examined.
          \end{itemize}
    \item End.
\end{itemize}

\subsubsection*{Analysis:}

\begin{itemize}
      \item $b$ - branching factor
      \item $d$ - depth of the graph
      \item \textbf{Time Complexity:} $O(b^d)$
      \item \textbf{Time Complexity:} $O(b^d)$
      \item Complete search
\end{itemize}